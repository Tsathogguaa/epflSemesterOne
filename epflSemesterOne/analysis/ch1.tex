  
\documentclass{article}
\usepackage[utf8]{inputenc}
\usepackage{amsmath}
\usepackage{tcolorbox}
\usepackage{amssymb}
\usepackage{amsthm}
\usepackage[
top    = 2.50cm,
bottom = 2.50cm,
left   = 2.75cm,
right  = 2.75cm]{geometry}
\usepackage{fancyhdr}
\pagestyle{fancy}
\lhead{Analysis 1}
\rhead{EPFL/Alp Ozen}

\newtheorem{thm}{Theorem}


\title{Analysis}
\author{SemesterOne analysis at EPFL}
\date{\vspace{-5ex}}
\newtheorem{theorem}{Theorem}[section]

\begin{document}

\maketitle

\section{Proofs}
\subsection{Some general proofs}
A valid proof is set of lines where each line logically follows from the next. 
A most famous proof is that $\sqrt{2} $ is irrational. 

\begin{tcolorbox}
\begin{proof}
	Suppose that $\sqrt{2} = \frac{a}{b}$ where $a,b \in \mathbb{Z}$ and $gcd(a,b) = 1$
	\\
	Now we have that $\sqrt{2}b = a$ which means that $2b^2 = a^2$.
	As result, $2 \vert a^2$ hence also $2 \vert a$. Thus we get that $a = 2k$ which also means that $b^2 = 2k^2$ hence $2 \vert b$. As result, $gcd(a,b) = 2$ which is a contradiction. Therefore, $\sqrt{2}$ must be irrational. 
	\end{proof}
\end{tcolorbox}

Quite interestingly, we can also construct a 'wrong' proof just through one fallacious assumption and a set of correct steps. 

\begin{tcolorbox}
    \textbf{Claim:} 1 is the largest integer.
    \\
    Proof: 
    \\
    Let $n$ be the largest integer. Then we have $n \geq n^2$. Which also means $0 \geq n^2 - n = n(n-1)$. Now we have that either $n < 0$ or $n - 1 < 0$. But we know that $ n \nless 0$ as $n$ is at least 1. Hence, $n - 1 < 0$ giving us the result $n < 1$ proving our theorem. Note that the mistake here is solely the assumption we made at the start that there was a largest integer. 
\end{tcolorbox}
\subsection{Proofs relating to infinite processes}
Consider the claim that $0.999\ldots = 1$.
One way to prove this claim, rather naively is this.

\begin{align*}
    9 \times 0.999\ldots \\
    = (10 - 1) \times 0.999\ldots\\
    = 9.999\ldots - 0.999\ldots 
    = 1
\end{align*}

Now a more formal proof is to use an infinite sum and limits. Here it is. 

\begin{tcolorbox}
\textbf{Analysis proof of 0.999\ldots = 1}
    \begin{align*}
        0.999\ldots = 9\lim_{k\to\infty}\sum_{i=1}^{k} (10^{-k})\\
        \lim_{k\to\infty}\sum_{i=1}^{k}(10^{-k}) = \frac{10^{-1} - 10^{-(k+1)}}{1-10^{-1}}\\
        = 9 \times \frac{1}{10} \times \frac{10}{9}\\
        = 1
    \end{align*}
\end{tcolorbox}

\subsection{Basic notions of sets}
The breakdown of sets used in 'standard' analysis are $\mathbb{N} \subseteq \mathbb{Z} \subseteq \mathbb{Q} \subseteq \mathbb{R}$.
\\
There are also some common set related notation that must be known. 
\begin{itemize}
    \item a \textbf{subset} $a \subseteq b$ is defined as $\{x \in b \vert \text{"condition"}\}$
    \item a \textbf{open interval} is defined as $\left]a,b[; r\in A, a < r < b$
    \item an \textbf{open ball} $B(a,\lambda) = \left]a-\lambda,a+\lambda[$
\end{itemize}

\subsection{The Reals}

The reals, denoted \mathbb{R} are an ordered field. Here is a more precise definition.

\begin{tcolorbox}
The reals are a set that have the 3 following axioms:
\begin{itemize}
    \item \mathbb{R} is an abelian group under (+) and \mathbb{R*} is an abelian group under ($\times$). In addition to this, multiplication distributes over addition.
    \item The order relation $\leq$ holds $\forall x \in \mathbb{R}$ That is:
    \begin{align*}
        x \leq y \otimes y \leq x\\
        x \leq y, y \leq x \implies x = y\\
        x \leq y \implies \forall a \in \mathbb{R},\  x + a \leq y + a\\
        0 \leq x, 0 \leq y \implies 0 \leq xy
    \end{align*}
    \item The inf and sup axioms hold
\end{itemize}

We shall now come the \textit{inf} and \textbf{sup} axioms. It should be intuitively clear that any subset of \mathbb{R} 
\end{tcolorbox}









\end{document} 
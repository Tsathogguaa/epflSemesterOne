  
\documentclass{article}
\usepackage[utf8]{inputenc}
\usepackage{amsmath}
\usepackage{tcolorbox}
\usepackage{amssymb}
\usepackage{amsthm}
\usepackage[
top    = 2.50cm,
bottom = 2.50cm,
left   = 2.75cm,
right  = 2.75cm]{geometry}
\usepackage{fancyhdr}
\pagestyle{fancy}
\lhead{Analysis 1}
\rhead{EPFL/Alp Ozen}

\newtheorem{thm}{Theorem}
\newtheorem{remark}{Remark}
\newtheorem{theorem}{Theorem}
\newtheorem{prop}{[Proposition]}
\newtheorem{definition}{Definition}

\title{Analysis}
\author{SemesterOne analysis at EPFL}
\date{\vspace{-5ex}}
\newtheorem{example}{Example}[section]
\newtheorem{axiom}{Axiom}
\newtheorem{cor}{Corollary}

\begin{document}

\maketitle

\section{Proofs and the reals}
\subsection{Some general proofs}
A valid proof is set of lines where each line logically follows from the next. 
A most famous proof is that $\sqrt{2} $ is irrational. 

\begin{tcolorbox}
\begin{proof}
	Suppose that $\sqrt{2} = \frac{a}{b}$ where $a,b \in \mathbb{Z}$ and $gcd(a,b) = 1$
	\\
	Now we have that $\sqrt{2}b = a$ which means that $2b^2 = a^2$.
	As result, $2 \vert a^2$ hence also $2 \vert a$. Thus we get that $a = 2k$ which also means that $b^2 = 2k^2$ hence $2 \vert b$. As result, $gcd(a,b) = 2$ which is a contradiction. Therefore, $\sqrt{2}$ must be irrational. 
	\end{proof}
\end{tcolorbox}

Quite interestingly, we can also construct a 'wrong' proof just through one fallacious assumption and a set of correct steps. 

\begin{tcolorbox}
    \textbf{Claim:} 1 is the largest integer.
    \\
    Proof: 
    \\
    Let $n$ be the largest integer. Then we have $n \geq n^2$. Which also means $0 \geq n^2 - n = n(n-1)$. Now we have that either $n < 0$ or $n - 1 < 0$. But we know that $ n \nless 0$ as $n$ is at least 1. Hence, $n - 1 < 0$ giving us the result $n < 1$ proving our theorem. Note that the mistake here is solely the assumption we made at the start that there was a largest integer. 
\end{tcolorbox}
\subsection{Proofs relating to infinite processes}
Consider the claim that $0.999\ldots = 1$.
One way to prove this claim, rather naively is this.

\begin{align*}
    9 \times 0.999\ldots \\
    = (10 - 1) \times 0.999\ldots\\
    = 9.999\ldots - 0.999\ldots 
    = 1
\end{align*}

Now a more formal proof is to use an infinite sum and limits. Here it is. 

\begin{tcolorbox}
\textbf{Analysis proof of 0.999\ldots = 1}
    \begin{align*}
        0.999\ldots = 9\lim_{k\to\infty}\sum_{i=1}^{k} (10^{-k})\\
        \lim_{k\to\infty}\sum_{i=1}^{k}(10^{-k}) = \frac{10^{-1} - 10^{-(k+1)}}{1-10^{-1}}\\
        = 9 \times \frac{1}{10} \times \frac{10}{9}\\
        = 1
    \end{align*}
\end{tcolorbox}

\subsection{Basic notions of sets}
The breakdown of sets used in 'standard' analysis are $\mathbb{N} \subseteq \mathbb{Z} \subseteq \mathbb{Q} \subseteq \mathbb{R}$.
\\
There are also some common set related notation that must be known. 
\begin{itemize}
    \item a \textbf{subset} $a \subseteq b$ is defined as $\{x \in b \vert \text{"condition"}\}$
    \item a \textbf{open interval} is defined as $\left]a,b[; r\in A, a < r < b$
    \item an \textbf{open ball} $B(a,\lambda) = \left]a-\lambda,a+\lambda[$
\end{itemize}

\subsection{The Reals}

The reals, denoted \mathbb{R} are an ordered field. Here is a more precise definition.

\begin{tcolorbox}
The reals are a set that have the 3 following axioms:
\begin{itemize}
    \item \mathbb{R} is an abelian group under (+) and \mathbb{R*} is an abelian group under ($\times$). In addition to this, multiplication distributes over addition.
    \item The order relation $\leq$ holds $\forall x \in \mathbb{R}$ That is:
    \begin{align*}
        x \leq y \otimes y \leq x\\
        x \leq y, y \leq x \implies x = y\\
        x \leq y \implies \forall a \in \mathbb{R},\  x + a \leq y + a\\
        0 \leq x, 0 \leq y \implies 0 \leq xy
    \end{align*}
    \item The inf and sup axioms hold
\end{itemize}

We shall now come the \textbf{inf} and \textbf{sup} axioms. It should be intuitively clear that any subset of \mathbb{R} 
\end{tcolorbox}

\subsection{Bounds}
Take some subset S in \mathbb{R}. An element B is called an upper bound of S if $\forall x \in S, B \geq x$. Similarly, it is a lower bound of S if $\forall x \in S, B \leq x$.
\\
The maximum B of a set S denoted $max(S)$ is such that $B \in S, \forall x \in S B \geq x$. 
\\
The supremum of a set S(if it exists) is the lowest upper bound. That is $sup(S) = b$ is such that,


    \begin{align}
        \forall x \in S, b \geq x\\
        \forall \epsilon > 0, \exists x_{\epsilon}, b - x_{\epsilon} \leq \epsilon
    \end{align}
    
    
    
    
    
\begin{remark}

   In our above definition, b does not have to be in S.

\end{remark}


\begin{remark}
    Condition 1 states that b is an upper bound of S. 
 
\end{remark}


\begin{remark}
       Given condition 1, b is the \textit{minimum} of the upper bounds of S.
\end{remark}
    
Some examples of $sup$ and $inf$

\begin{example}
\begin{align*}
 Sup ]a,b[ = b\\
 Inf ]a,b[ = a\\
 Sup \{ x\in \mathbb{R} \vert x = 2k\} \implies \text{Sup doesn't exist.}
\end{align*}
\end{example}


We now establish the infinimum axiom.
\begin{axiom}
All non-empty subsets of $\mathbb{R}_{+}^{*}$ have a highest lower boundary(aka.infinimum)
\end{axiom}

\subsection{\mathbb{Q} is dense in \mathbb{R}}
We claim that between every real number, one can find a rational number. Here's the proof,

\begin{tcolorbox}
\begin{proof}
    Let $x<y \in \mathbb{R}$ Suppose now that $\exists a \in \mathbb{Q}$ such that $x<a<y$. By the Archimedean principle(there is always a greater natural number, $n > \frac{1}{y-x}$ which implies $ny>nx+1$ Now since $ny>nx+1$, there is guaranteed to be some integer in the open bound $]nx,ny[$ which we denote $P$. Dividing by $n$, we get that $\frac{P}{n} \in ]x,y[$ which proves the theorem. 
    \tag*{\qedhere}
\end{proof}
\end{tcolorbox}

\subsection{Integer and fractional part}

Any number $\in \mathbb{R}$ has a integer and fractional part(at least intuitively). Let's formally define these. For some $x \in \mathbb{R}$, let $S := \{n\in\mathbb{N} \vert n > x\}$ Now since $S$ is bounded from below, letting $N$ be the minimum of this set, we obtain that $N \not\in S$. N-1 is thus called the integer part of x denoted [x]. ie. [6.4] = 6
\\
Similarly, the fractional part of x denoted {x} is simply ${x} = x - [x]$

\subsection{Pinning it down:Sup/Inf, bounds, max/min}
\begin{definition}
for a given set $S \subseteq \mathbb{R}$, we have the following:
\\
\textbf{Sup s = b} $\iff \forall \epsilon > 0, \exists x_{\epsilon} \in S, s.t. b - x_{\epsilon} < \epsilon$ (resp. Inf s has the flipped argument)
\\
\textbf{Upper bound = b} $\iff \forall x \in S, b\geq x$(resp. lower bound)
\\
\textbf{Max s = b} $\iff$ b is an upper bound and $b \in S$

\end{definition}
\\

And for the sake of repeating the early axiom(but very important) the infimum axiom is:
\begin{axiom}
For all non-empty subsets of $\mathbb{R}$, $inf S$ exists. 
\end{axiom}

Now we make the first claim in this course that uses an epsilon proof.
\begin{tcolorbox}

\begin{prop}
Whenever $S \subseteq \mathbb{N}$, then $inf S = min S$
\end{prop}

\begin{proof}
    Now, by our axiom, we have that $\mathbb{N} \subseteq \mathbb{R}$ hence we know that $inf S$ exists. We now have to show that $inf S = min S$. 
    \\
    
    Suppose that $inf S \not =  min S$ and let $inf S = b$. Now clearly, $b+\epsilon$ is not a lower bound of S. Now, let $\epsilon = \frac{1}{2}$. Because, $b + \epsilon$ is not a lower bound, we know that $\exists s_{e} < b + \epsilon$. 
    \\
    
    Now $s_{e$ is also not a lower bound, so let's pick $\epsilon^{\prime\prime} = s_{\epsilon} - d$. Now again, $s_{\epsilon^{\prime\prime}}$ must exist. We obtain yet the following:
    \\
    \begin{equation*}
        d < s_{\epsilon^{\prime\prime}} < s_{\epsilon} < d + 1/2 
    \end{equation*}
    
    Now, two natural numbers clearly can not be in an interval which is only $\frac{1}{2}$ units long. Hence, contradiction which means that $d \in S$
    \tag*{\qedhere}
\end{proof}

\end{tcolorbox}
\\
Let's now prove that $\sqrt{2}$ belongs to the reals. For this, we need the following corollary and axiom. 

\begin{cor}
Every non-empty subset of \mathbb{R} with an upper boundary admits a supremum. 
\end{cor}

\begin{axiom}
An ordered field F, which \mathbb{R} is, has the Archimedean property if given any positive x and y in F, $\exists n \in \mathbb{Z}$ s.t. $nx > y$
\end{axiom}

\begin{tcolorbox}

\begin{prop}
$\sqrt{2} \in \mathbb{R}$
\end{prop}

\begin{proof}
Suppose we define a set $S = \{r \in \mathbb{R} \vert r \geq 0, r^{2} < 2\}$. Now, as $S$ is a non-empty subset of $\mathbb{R}$ bounded from above(i.e. 2 is an upper bound) we know that $sup S = x$ exists. Our goal is to show that both $x^{2} < 2$ and $x^{2} > 2$ lead to a contradiction. 
\\

\textbf{Case 1:} Suppose $x^{2} < 2$. We want to find $x + \frac{1}{2} \in S$ which implies that $x$ is not an upper bound as $x < x + \frac{1}{2}$ 
\\
$$(x+\frac{1}{2})^2 = x^2 \frac{2}{x} + \frac{1}{n^2} \leq x^2 + \frac{2}{x} + \frac{1}{n} = x^2 + \frac{1}{n}(2x + 1) $$
\\
Now, we want to show that we can pick an $n$ s.t. $ x^2 + \frac{1}{n}(2x + 1) < 2$. If we can pick such an n, then we know by transitivity of $<$ that $(x+\frac{1}{n})^2 < 2$ as  $x^2 \frac{2}{x} + \frac{1}{n^2} \leq x^2 + \frac{2}{x} + \frac{1}{n}$
   
    \tag*{\qedhere}
\end{proof}

\end{tcolorbox}





\end{document} 
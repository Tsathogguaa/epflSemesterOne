\documentclass{article}
\usepackage[utf8]{inputenc}
\usepackage{amsmath}
\usepackage{tcolorbox}
\usepackage{amssymb}
\usepackage[
top    = 2.50cm,
bottom = 2.50cm,
left   = 2.75cm,
right  = 2.75cm]{geometry}
\usepackage{fancyhdr}
\pagestyle{fancy}
\lhead{Linear algebra basics}
\rhead{EPFL/Alp Ozen}

\newtheorem{thm}{Theorem}


\title{Linear Algebra 1}
\author{alp.ozen}
\date{September 2019}
\newtheorem{theorem}{Theorem}[section]

\begin{document}

\maketitle

\section{Linear equations}
\subsection{Basics}

\begin{tcolorbox}
A linear equation is any equation of form:
$a_{1}x_{1}+ ..... + a_{n}x_{n} = b$
where the a are 'scalars' that belong to a field and the x belong to the vector set. 
\end{tcolorbox}

A \textit{system of linear equation}s is simply a collection of linear equations. The solution of this system, if there is any, is an ordered list $(s_{1},...,s_{n})$ where each $s_{i}$ is the value of each $x_{i}$.
\\

A system consisting of 2 unknowns and two linear equations is generally the intersection of two lines on a cartesian plane. Note that the lines may be parallel or even colinear. 
\\

In general, a system will have:
\begin{itemize}
    \item no solution
    \item unique solution
    \item infinite solutions 
\end{itemize}
\\

We may choose to represent a system of linear equations as an \textit{augmented matrix}. A matrix is called n$\times$m if it is of form:
\begin{equation*}
    \begin{bmatrix}
    a_{11} && a{12} &&  ... \\
    a_{21} && a {22}  && ... \\
    \end{bmatrix}
\end{equation*}

\subsection{Solving a linear system}

When solving a system, our goal is to replace each linear equation with an equivalent set(one that has the same solution) and obtain single linear equations which are trivial to solve. 
\\

In solving a system, we use the elementary row operations which are: 

\begin{tcolorbox}
\begin{itemize}
    \item Interchange two rows (or columns).
    \item Multiply each element in a row (or column) by a non-zero number.
    \item Multiply a row (or column) by a non-zero number and add the result to another row (or column).
\end{itemize}
\end{tcolorbox}

Our goal is to transform our matrix into echelon or row reduced echelon form. A matrix is in echelon form if it looks like this: \begin{equation*}
\begin{bmatrix}
    \blacktriangledown &&  \blacktriangle && \blacktriangle \\
    0 && \blacktriangledown && \blacktriangle \\
    0 && 0 && \blacktriangledown

\end{bmatrix}
    
\end{equation*}

Here, each $\blacktriangledown$ and $\blacktriangle$ may take on any value from the set the vector space is defined on. 
\\

To obtain this form, we first arrange our matrix into a form where the row with the least amount of trailing zeros is placed uptop. Then we ensure the \textit{pivot position}(meaning first non-zero entry) has only 0 in its own column. When done, we move on the second row, find the pivot position and repeat. We repeat this process for all rows. 
\\

If we have a system where the number of unknowns exceeds the number of equations, we obtain a parametric solution. Consider this: 

\begin{example}
\begin{equation*}
  \begin{bmatrix}
    1 && 0 &&  -5 && 1 \\
    0 && 1 && 1 && 4 \\
    0 && 0  && 0 && 0 \\
    \end{bmatrix}
\end{equation*}

Which means: 
$$ x_{1} - 5x_{3} = 1$$
$$ x_{2} +  x_{3} = 4$$

Now, we can express both $x_{1}$ and $x_{2}$ in terms of $x_{3}$. We call $x_{1}$ and $x_{2}$ basic variables and $x_{3}$ the free variable. We call $x_{3}$ a free variable as we are free to choose any value for it. 
\end{example}

Most importantly, these are the conditions for row echelon and reduced row echelon form. 

\begin{tcolorbox}
 \textbf{Row echelon form if:}
 \begin{itemize}
     \item all nonzero rows are above zero rows
     \item a leading entry is the right column of a leading entry above it
     \item all entries in a column below a leading entry are 0
 \end{itemize}
 \textbf{Reduced row echelon also if:}
 \begin{itemize}
     \item leading entry in each nonzero row is 1
     \item the entries in the column of the leading entry are 0
 \end{itemize}
\end{tcolorbox}
\\ 

To obtain reduced row echelon form, starting from the lowest row, the terms in the column with the leading term are made zero and the leading term scaled to 1. We repeat this process. 
\\

As a general remark, we obtain the following.

\begin{thm}
A system is consistent iff, the rightmost column has no pivot. More simply, iff the lowest row is not of the form: 
\begin{bmatrix}
    0 && 0 \dots && b
\end{bmatrix}
where b is nonzero. 
\end{thm}


















\end{document}
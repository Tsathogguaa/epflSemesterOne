\documentclass{article}
\usepackage[utf8]{inputenc}
\usepackage{amsmath}
\usepackage{tcolorbox}

\addtolength{\topmargin}{-1.0in}


\title{Linear Algebra 1}
\author{alp.ozen}
\date{September 2019}

\begin{document}

\maketitle

\section{Linear equations}
\subsection{Basics}

\begin{tcolorbox}
A linear equation is any equation of form:
$a_{1}x_{1}+ ..... + a_{n}x_{n} = b$
where the a are 'scalars' that belong to a field and the x belong to the vector set. 
\end{tcolorbox}

A \textit{system of linear equation}s is simply a collection of linear equations. The solution of this system, if there is any, is an ordered list $(s_{1},...,s_{n})$ where each $s_{i}$ is the value of each $x_{i}$.
\\

A system consisting of 2 unknowns and two linear equations is generally the intersection of two lines on a cartesian plane. Note that the lines may be parallel or even colinear. 
\\

In general, a system will have:
\begin{itemize}
    \item no solution
    \item unique solution
    \item infinite solutions 
\end{itemize}
\\

We may choose to represent a system of linear equations as an \textit{augmented matrix}. A matrix is called n$x$m if it is of form:
\begin{equation*}
    \begin{bmatrix}
    a_{11} && a{12} &&  ... \\
    a_{21} && a {22}  && ... \\
    \end{bmatrix}
\end{equation*}

\subsection{Solving a linear system}

When solving a system, our goal is to replace each linear equation with an equivalent set(one that has the same solution) and obtain single linear equations which are trivial to solve. 
\\

In solving a system, we use the elementary row operations which are: 

\begin{tcolorbox}
\begin{itemize}
    \item Interchange two rows (or columns).
    \item Multiply each element in a row (or column) by a non-zero number.
    \item Multiply a row (or column) by a non-zero number and add the result to another row (or column).
\end{itemize}
\end{tcolorbox}









\end{document}